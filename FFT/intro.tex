\section{Introduction}
% %<<FF>> ***********************************************************************
 \begin{frame}
 \frametitle{DFT / FFT Objectives}
{\bf General objectives of the \textcolor{blue}{discrete} fourier transformation}
\begin{itemize}
  \item Analyzing a time domain signa in the frequncy domain $\leadsto$ Fourier Transform
   \item The fast fouriet transform (FFT) is an efficient method to numerically compute a dicrete fourier transform (DFT)
   % \item Mein FFT Arbeitsverzeichnis f\"ur die Simulationen liegt % \\
   %  % \begin{verbatim}
   %  %     \texttt{C:/workspace/1027\_pJoint\_squeakVEAE/SCALEXIO/myDevelop}
   % \texttt{C:/workspace/teachingSMC/FFT/Matlab/}     
   %  %\end{verbatim}
   %  %    bzw. \texttt{FFT4dSpace.slx} und Postprocessing \texttt{VisFFTsimResults.m}
   % bzw. \texttt{FFT4dSpace\_rtKolloq.slx} und Postprocessing \texttt{VisFFTsimResults\_rtKolloq.m}

   
  \item     The goal is to analyze to following signal:

 \begin{eqnarray*}
 %  \boxed{
   x(t) &=& c+ a_1 \sin(\omega_1 t) + a_2 \cos(\omega_2 t);
   % }
  \\          
   \text{discrete time}
   \\
  \textcolor{blue}{
   x[n]}       &\textcolor{blue}{=}&  \textcolor{blue}{c + a_1 \sin(\omega_1 n \ind{T}{s}) + a_2 \sin(\omega_2 n \ind{T}{s}), \quad n=0,1,2,\ldots
    }
  \end{eqnarray*}
with sample time $\ind{T}{s}$ or sample frequency $\ind{f}{s} = \frac{1}{\ind{T}{s}}$ .
\end{itemize}
% \pause % *********************************************************************
% {\bf Features for this class of control?}
% \begin{itemize}
%  \item Discontinuous control law
%  \item For standard sliding mode (first order): chattering effect, robustness
%  \item For higher order sliding mode: accuracy, finite time convergence, robustness
% \end{itemize}
% \pause % *********************************************************************
% \begin{block}{Remark} {\bf Sliding mode} as a phenomenon may appear in
%   a dynamic system governed by ordinary differential equation with
%   {\em discontinuous right hand side}
% \end{block}

\end{frame}


% <<FF>> ************************************************************
\begin{frame}
  \frametitle{Overview}
  \begin{itemize}
  \item Objectives: time domain to frequency domain
  \item time signal - discrete time signal - DFT - FFT
  \item Sample Time / Number of  Samples / Bin Width / Nyquist Frequency / Leackage effect
  \item Simulink Simulation
  \item dSpace Implementation in Real Time
   
  \end{itemize}
\end{frame}
% *****************************************************************


%%% Local Variables:
%%% mode: latex
%%% TeX-master: "FFT4Students"
%%% End:
